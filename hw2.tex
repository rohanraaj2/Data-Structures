\documentclass[addpoints]{exam}

\usepackage{caption}
\usepackage{graphbox}
\usepackage{hyperref}
\usepackage{multirow}
\usepackage{pythonhighlight}
\usepackage{ragged2e}
\usepackage{subcaption}
\usepackage{tabularx}
\usepackage{titling}
\usepackage{xcolor}
\usepackage{float}

% Header and footer.
\pagestyle{headandfoot}
\runningheadrule
\runningfootrule
\runningheader{CS 201, Spring 2023}{HW 2: Skiplists and Dynamic List ADT}{\theauthor}
\runningfooter{}{Page \thepage\ of \numpages}{}
\firstpageheader{}{}{}

\qformat{{\large\bf \thequestion. \thequestiontitle}\hfill[\totalpoints\ points]}
% \qformat{{\large\bf \thequestion. \thequestiontitle}\hfill}
\boxedpoints

\printanswers

\graphicspath{{images/}}

\newcommand\colheader[1]{\multicolumn{1}{c}{#1}} % Note: no vertical bars

\title{Homework 2: Skiplists and Dynamic List ADT}
\author{l2-group-6}  % <=== Enter your team name.
\date{Habib University\\Spring 2023}

\begin{document}
\begin{questions}

  \titledquestion{A Ranked Set}[10]\footnote{Adapted from Exercise 4.9 in the textbook.}
  Design a version of a skiplist that implements the \texttt{SSet} interface, but also allows fast access to elements by \textit{rank}. That is, it also supports the function \texttt{get(i)}, which returns the element whose rank is \texttt{i} in $O(\log n)$ expected time. (The rank of an element \texttt{x} in an \texttt{SSet} is the number of elements in the \texttt{SSet} that are less than \texttt{x}.)
  Describe how your version differs from a regular skiplist and provide pseudocode of \texttt{find(x)} and \texttt{get(i)} for this version.
  \begin{solution}
    In our version of the skiplist, we are keeping track of additional information (rank, span) alongside the other information contained in a skiplist such as head, forward, height etc. Due to this change, we now have two additional functions, $\text{get}(i)$ and $\text{find}(x)$, which allows us to access elements faster using their ranks. This also means that now, when updating the skiplist (adding/removing nodes), we will have to update the ranks of each node.
    \par The pseudocode for the $\text{get}(i)$ and $\text{find}(x)$ functions are as follows, taking the assumption that rank and span have already been initialized in the skiplist:
    \begin{lstlisting}[language=Python]
    def find(self, x) -> int:
      current_node = self.head
      # start from the highest level
      for level in range(self.height() - 1, -1, -1):         
          # check if there is another node on the same level and key < x 
          while current_node.forward[level] != None           
          and current_node.forward[level].key < x:
              # if true, assign the next node to current node             
              current_node = current_node.forward[level]   
      # assign next node at level 0 to current node           
      current_node = current_node.forward[0]                  
      if current_node != None and current_node.key == x:
          # return value of current node
          return current_node.value                           
  
    def get(self, i) -> int:
      current_node = self.head
      rank = -1
      # start from the highest level
      for level in range(self.height()-1, -1, -1):
          # check if there is another node on the same level and rank+span < i           
          while current_node.forward[level] != None         
          and rank + current_node.forward[level].span <= i: 
              # if true, update the rank and assign the next node to current node
              rank += current_node.forward[level].span        
              current_node = current_node.forward[level]     
      # assign next node at level 0 to current node
      current_node = current_node.forward[0]                  
      if current_node != None and rank + 1 == i:
          # returns element with rank i
          return current_node.value                           
      \end{lstlisting}
  \end{solution}

  \titledquestion{Finger Search}[10]\footnote{Adapted from Exercise 4.10 in the textbook.}
  A \textit{finger} in a skiplist is an array that stores the sequence of nodes on a search path at which the search path goes down. (The variable \texttt{stack} in the \texttt{add(x)} code on page 87 is a finger; the shaded nodes in Figure 4.3 show the contents of the finger.) One can think of a finger as pointing out the path to a node in the lowest list, $L_0$.

  A \textit{finger search} implements the find(x) operation using a finger, walking up the list using the finger until reaching a node \texttt{u} such that \texttt{u.x < x} and \texttt{u.next = nil} or \texttt{u.next.x > x} and then performing a normal search for \texttt{x} starting from \texttt{u}. It is possible to prove that the expected number of steps required for a finger search is $O(1+\log r)$, where $r$ is the number values in $L_0$ between \texttt{x} and the value pointed to by the finger.

  Design, i.e. provide the necessary pseudo code for, a version of a skiplist that implements \texttt{find(x)} operations using an internal finger. This subclass stores a finger, which is then used so that every \texttt{find(x)} operation is implemented as a finger search. During each \texttt{find(x)} operation the finger is updated so that each \texttt{find(x)} operation uses, as a starting point, a finger that points to the result of the previous \texttt{find(x)} operation.
  \begin{solution}
    \begin{lstlisting}[language=Python]
      class SkipList(object):
          def __init__(self) -> None:
              # initialize finger to empty array
              self.finger = []       
              self.head = Node((None, None))
              self.max_level = 0
              self.size = 0
      
          def find(self, x) -> int:
              # check if array is empty
              if len(self.finger) == 0:
                  # if true, start search from head
                  node = self.head
                  level = self.maxLevel
              else:
                  # else, start search from finger node
                  node, level = self.finger[-1]
      
              while True:
                  if level == 0:
                      # element not found
                      return None
                  elif node.next[level] != None and node.next[level].x < x:
                      node = node.next[level]
                  elif node.next[level] != None and node.next[level].x == x:
                      # update finger to point to this node
                      self.finger.append((node, level))
                      return node.next[level]
                  else:
                      # node.next[level] != None and node.next[level].x > x
                      # move down a level and update finger
                      level -= 1
                      self.finger.append((node, level))
          \end{lstlisting}
  \end{solution}


\end{questions}


\end{document}
%%% Local Variables:
%%% mode: latex
%%% TeX-master: t
%%% End:
